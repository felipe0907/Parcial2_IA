\documentclass[12pt,a4paper]{article}
\usepackage{fontspec}
\usepackage{polyglossia}
\setmainlanguage{spanish}
\setmainfont{Latin Modern Roman}
\usepackage{enumitem}
\usepackage{hyperref}
\usepackage{geometry}
\geometry{margin=1in}


\begin{document}

\section*{Parcial 2}

\subsection*{Predicción de Ingresos con Datos del Censo}

\textbf{Contexto}

En este proyecto analizarás el dataset Adult Census Income del repositorio UCI Machine Learning,
que contiene datos del censo estadounidense de 1994. El objetivo es desarrollar una red neuronal
que prediga si una persona gana más de \$50{,}000 anuales basándose en caracteŕisticas demográficas y
socioeconómicas.

\subsection*{Reglas Generales}

\begin{itemize}
\item Puedes utilizar recursos de internet como referencia, pero debes comprender el código, no solo copiarlo
\item Utiliza PyTorch obligatoriamente para implementar redes neuronales
\item Documenta tu código y decisiones técnicas
\end{itemize}

\section{Recolección y procesamiento de datos (10\%)}

\begin{enumerate}
\item Diríjase a la URL \href{https://archive.ics.uci.edu/dataset/2/adult}{https://archive.ics.uci.edu/dataset/2/adult} y descargue el dataset. Cargue los archivos importantes (datos de entrenamiento y de prueba).

Estos pueden ser descargados en Python con el siguiente código:

from ucimlrepo import fetch_ucirepo 
  
# fetch dataset 
adult = fetch_ucirepo(id=2) 
  
# data (as pandas dataframes) 
X = adult.data.features 
y = adult.data.targets 
  
# metadata 
print(adult.metadata) 
  
# variable information 
print(adult.variables) 


\item Con los datos de prueba, haga un split 50/50 para crear los datos de validación.
\item Realice Exploratory Data Analysis (EDA).
\item Haga un procesamiento de los datos para sus modelos. Cuidado con el Data Leakage
\end{enumerate}

\section{Desarrollo de algoritmos (70\%)}

\subsection{Modelo Baseline (10\%)}
\begin{enumerate}
\item Realice un modelo de regresión logística y entrénelo.
\item Obtenga las métricas (las que se usan para un problema de clasificación binario) de entrenamiento/validación/prueba.
\end{enumerate}

\subsection{Modelo de Redes Neuronales (60\%)}
\begin{enumerate}
\item Cree la arquitectura base de la red (MLP)
\item Defina la función de pérdida y el optimizador.
\item Cree un loop de entrenamiento que incluya la validación.
\item Realice al menos 5 experimentos con distintas configuraciones de hiperparámetros. Utilice la GPU. Sea generoso con el número de épocas, capas y neuronas. Haga las gráficas de pérdida vs épocas. Analice las gráficas y detecte si hay overfitting/underfitting.
\item Investigue en internet e implemente las siguientes técnicas: Dropout y EarlyStopping. Pista: Dropout se implementa dentro de la arquitectura base de la red. EarlyStopping se usa dentro del loop de entrenamiento.
\item Vuelva a realizar (4) y obtenga el mejor MLP con regularización.
\item Obtenga las métricas (las que se usan para un problema de clasificación binario) de entrenamiento/validación/prueba.
\end{enumerate}

\section{Reporte y GitHub (20\%)}

\begin{enumerate}
\item Cree un reporte en un archivo txt con lo siguiente:
\begin{itemize}
  \item Decisiones del procesamiento de datos (Si usaron todas las características, si crearon unas nuevas, el tipo de transformaciones que hicieron, etc.)
  \item Hiperparámetros del mejor experimento de MLP. Apóyese de las gráficas generadas también. Compare los mejores MLP sin regularización y con regularización.
  \item Compare el mejor MLP y la regresión lineal a partir de sus métricas. Interprete los resultados.
\end{itemize}
\item El codigo se debe correr de inicio a fin y al finalizar debe dar los resultados del mejor MLP. Si los resultados no tienen nada de similitud, se les bajará puntos.
\item Si el codigo genera algún error de inicio a fin y no se pueden generar los resultados del reporte, se bajarán puntos.
\end{enumerate}

\bigskip
\noindent
\textbf{Resumen de ponderación:}
\begin{itemize}[label=--]
\item Recolección y procesamiento de datos (10\%)
\item Desarrollo de algoritmos (70\%)
\item Modelo Baseline (10\%)
\item Modelo de Redes Neuronales (60\%)
\item Reporte y GitHub (20\%)
\end{itemize}

\end{document}
